\documentclass[12pt,a4paper]{report}
\usepackage[utf8]{inputenc}
\usepackage[english]{babel}
\usepackage[T1]{fontenc}
\usepackage{amsmath}
\usepackage{amsfonts}
\usepackage{amssymb}
\usepackage[left=2cm,right=2cm,top=2cm,bottom=2cm]{geometry}
\author{Justin Dekeyser}
\title{The Family Calendar}

\setlength{\parindent}{0em}
\setlength{\parskip}{1em}
\linespread{1.5}

\usepackage{listings}
\usepackage{color}
\usepackage[pdftex]{xcolor}
\usepackage{textcomp}
\usepackage{caption}
\usepackage{hyperref}

\usepackage{amsthm}
\newtheorem{definition}{Definition}
\newtheorem{deduction}{Deduction}

\definecolor{jskeywords}{HTML}{0000FF}% JavaScript keywords
\definecolor{jsextkeywords}{HTML}{9900FF}% JavaScript extended keywords

\definecolor{identifiers}{HTML}{645452} % identfiers
\definecolor{string}{HTML}{B57281} % string literals
\definecolor{allcomment}{HTML}{808080} % comment

\definecolor{linenumber}{HTML}{996515} % line number
\definecolor{apricot}{HTML}{98777B} % numbers
\definecolor{linenofill}{HTML}{BEBEBE} % line number fill color
\definecolor{antiquefuchsia}{HTML}{915C83} % braces
\definecolor{ballblue}{HTML}{21ABCD} % braces

\definecolor{captioncolor}{rgb}{0.39, 0.33, 0.32} % caption color
\captionsetup[lstlisting]{font={color=captioncolor, small,tt}}

\captionsetup[lstlisting]{font={color=captioncolor, small, tt}}
\DeclareCaptionFormat{listing}{\rule{\dimexpr\textwidth+17pt\relax}{0.4pt}\vskip1pt#1#2#3}
\captionsetup[lstlisting]{format=listing,singlelinecheck=false, margin=0pt, font={sf},labelsep=space,labelfont=bf}

\lstdefinelanguage{JavaScript}{
    alsoletter={.},
    keywords={arguments,await,break,case,catch,class,const,continue,debugger,default,delete,do,else,enum,eval,export,extends,false,finally,for,function,if,implements,import,in,instanceof,interface,let,new,null,package,private,protected,public,return,static,super,switch,this,throw,true,try,typeof,var,void,while,with,yield}, % JavaScript ES6 keywords
    keywordstyle=\color{jskeywords}\bfseries,
    ndkeywords={add, apply, args, Array, Array.from, Array.isArray, Array.of , Array.prototype, ArrayBuffer, bind, Boolean, call, charAt, charCodeAt, clear, codePointAt, concat, constructor, copyWithin, DataView, Date, Date.now, Date.parse, Date.prototype, Date.UTC, decodeURI, decodeURIComponent, encodeURI, encodeURIComponent, endsWith, entries, Error, Error.prototype, EvalError, every, false, fill, filter, find, findIndex, Float32Array, Float64Array, forEach, FulfillPromise, Function, Function.length, get, getDate, getDay, getFullYear, getHours, getMilliseconds, getMinutes, getMonth, getSeconds, getTime, getTimezoneOffset, getUTCDate, getUTCDay, getUTCFullYear, getUTCHours, getUTCMilliseconds, getUTCMinutes, getUTCMonth, getUTCSeconds, has,hasInstance, hasOwnProperty, ignoreCase, includes, indexOf, indexOf, Infinity, Int8Array, Int16Array, Int32Array, isConcatSpreadable, isFinite, isNaN, IsPromise, isPrototypeOf, Iterable, iterator, join, JSON, JSON.parse, JSON.stringify, keys, lastIndexOf, lastIndexOf, length, localeCompare, map, Map, match, match, Math, Math.abs , Math.acos, Math.acosh, Math.asin, Math.asinh, Math.atan, Math.atan2, Math.atanh, Math.cbrt, Math.ceil, Math.clz32, Math.cos, Math.cosh,  Math.E, Math.exp, Math.expm1, Math.floor, Math.fround, Math.hypot, Math.imul, Math.LN2, Math.LN10, Math.log, Math.log1p, Math.log2, Math.LOG2E, Math.log10, Math.LOG10E, Math.max, Math.min, Math.PI, Math.pow, Math.random, Math.round, Math.sign, Math.sin, Math.sinh, Math.sqrt, Math.SQRT1_2, Math.SQRT2, Math.tan, Math.tanh, Math.trunc, message, multiline, name, NaN, NewPromiseCapability, next, normalize, null, Number, Number.EPSILON, Number.isFinite, Number.isInteger, Number.isNaN, Number.isSafeInteger, Number.MAX_SAFE_INTEGER, Number.MAX_VALUE, Number.MIN_SAFE_INTEGER, Number.MIN_VALUE, Number.NaN, Number.NEGATIVE_INFINITY, Number.parseFloat, Number.parseInt, Number.POSITIVE_INFINITY, Number.prototype, Object, Object, Object.assign, Object.create, Object.defineProperties, Object.defineProperty, Object.freeze, Object.getOwnPropertyDescriptor, Object.getOwnPropertyNames, Object.getOwnPropertySymbols, Object.getPrototypeOf, Object.is, Object.isExtensible, Object.isFrozen, Object.isSealed, Object.keys, Object.preventExtensions, Object.prototype, Object.seal, Object.setPrototypeOf, of, parseFloat, parseInt, pop, Promise, Promise.all , Promise.race, Promise.reject, Promise.resolve, PromiseReactionJob, propertyIsEnumerable, prototype, Proxy, Proxy.revocable , push, RangeError, reduce, reduceRight, ReferenceError, Reflect, Reflect.apply, Reflect.construct , Reflect.defineProperty, Reflect.deleteProperty, Reflect.enumerate, Reflect.get, Reflect.getOwnPropertyDescriptor, Reflect.getPrototypeOf, Reflect.has, Reflect.isExtensible, Reflect.ownKeys, Reflect.preventExtensions, Reflect.set, Reflect.setPrototypeOf, Reflection, RegExp, RegExp, RegExp.prototype, repeat, replace, replace, reverse, search, search, Set, set, setDate, setFullYear, setHours, setMilliseconds, setMinutes, setMonth, setSeconds, setTime, setUTCDate, setUTCFullYear, setUTCHours, setUTCMilliseconds, setUTCMinutes, setUTCMonth, setUTCSeconds, shift, slice, slice, some, sort, species, splice, split, split, startsWith, String, String.fromCharCode, String.fromCodePoint, String.raw, substring, Symbol, Symbol.for, Symbol.hasInstance, Symbol.isConcatSpreadable, Symbol.iterator, Symbol.keyFor, Symbol.match, Symbol.prototype, Symbol.replace, Symbol.replace, Symbol.search, Symbol.species, Symbol.split, Symbol.toPrimitive, Symbol.toStringTag, Symbol.unscopables, SyntaxError, then, toDateString, toExponential, toFixed, toISOString, toJSON, toLocaleDateString, toLocaleLowerCase, toLocaleString, toLocaleString, toLocaleString, toLocaleString, toLocaleTimeString, toLocaleUpperCase, toLowerCase, toPrecision, toPrimitive, toString, toStringTag, toTimeString, toUpperCase, toUTCString, TriggerPromiseReactions, trim, true, TypeError, Uint8Array, Uint8ClampedArray, Uint16Array, Uint32Array, undefined, unscopables, unshift, URIError, valueOf, WeakMap, WeakSet
    }, % JavaScript extended keywords
    ndkeywordstyle=\color{jsextkeywords}\bfseries,
    identifierstyle=\color{identifiers},
    sensitive=true,
    stringstyle=\color{string}\ttfamily,
    morestring=[b]",
    morestring=[d]',
    morestring=[s][\color{string}\ttfamily]{`}{`},
    commentstyle=\color{red}\itshape,
    morecomment=[l][\color{allcomment}]{//},
    morecomment=[s][\color{allcomment}]{/*}{*/},
    morecomment=[s][\color{allcomment}]{/**}{*/},
    emph={app.all, app.delete, app.disable, app.disabled, app.enable, app.enabled, app.engine, app.get, app.listen, app.locals, app.METHOD, app.mountpath, app.param, app.path, app.post, app.put, app.render, app.route, app.set, app.use, express, express.Router, express.static, req.acceptLanguages, req.accepts, req.acceptsCharsets, req.acceptsEncodings, req.app, req.baseUrl, req.body, req.cookies, req.fresh, req.get, req.hostname, req.ip, req.ips, req.is, req.method, req.originalUrl, req.param, req.params, req.path, req.protocol, req.query, req.range, req.route, req.secure, req.signedCookies, req.stale, req.subdomains, req.xhr, res.app, res.append, res.attachment, res.clearCookie, res.cookies, res.download, res.end, res.format, res.get, res.headersSent, res.json, res.jsonp, res.links, res.locals, res.location, res.redirect, res.render, res.sendFile, res.sendStatus, res.set, res.status, res.type, res.vary, router.all, router.METHOD, router.param, router.route, router.use}, % express keywords
    emph={[2]agent.createConnection, agent.destroy, agent.freeSockets, agent.getName, agent.maxFreeSockets, agent.maxSockets, agent.requests, agent.sockets, certificate.exportChallenge, certificate.exportPublicKey, certificate.verifySpkac, child.channel, child.connected, child.disconnect, child.kill, child.pid, child.send, child.stderr, child.stdin, child.stdio, child.stdout, child_process.exec, child_process.execFile, child_process.execFileSync, child_process.execSync, child_process.fork, child_process.spawn, child_process.spawnSync, cipher.final, cipher.getAuthTag, cipher.setAAD, cipher.setAutoPadding, cipher.update, clearImmediate, clearImmediate, clearInterval, clearInterval, clearTimeout, clearTimeout, console, console.assert, console.dir, console.error, console.info, console.log, console.time, console.timeEnd, console.trace, console.warn, decipher.final, decipher.setAAD, decipher.setAuthTag, decipher.setAutoPadding, decipher.update, dgram.createSocket, dgram.createSocket, diffieHellman.computeSecret, diffieHellman.generateKeys, diffieHellman.getGenerator, diffieHellman.getPrime, diffieHellman.getPrivateKey, diffieHellman.getPublicKey, diffieHellman.setPrivateKey, diffieHellman.setPublicKey, diffieHellman.verifyError, dns.getServers, dns.getServers, dns.lookup, dns.lookup, dns.lookupService, dns.resolve, dns.resolve4, dns.resolve6, dns.resolveCname, dns.resolveMx, dns.resolveNaptr, dns.resolveNs, dns.resolvePtr, dns.resolveSoa, dns.resolveSrv, dns.resolveTxt, dns.reverse, dns.setServers, ecdh.computeSecret, ecdh.generateKeys, ecdh.getPrivateKey, ecdh.getPublicKey, ecdh.setPrivateKey, ecdh.setPublicKey, error.address, error.code, error.errno, error.message, error.path, error.port, error.stack, error.syscall, exports, fs.access, fs.accessSync, fs.appendFile, fs.appendFileSync, fs.chmod, fs.chmodSync, fs.chown, fs.chownSync, fs.close, fs.closeSync, fs.constants, fs.createReadStream, fs.createWriteStream, fs.exists, global, http.createServer, http.get, http.globalAgent, http.request, https.createServer, https.get, https.globalAgent, https.request, message.destroy, message.headers, message.httpVersion, message.method, message.rawHeaders, message.rawTrailers, message.setTimeout, message.socket, message.statusCode, message.statusMessage, message.trailers, message.url, module, module.children, module.exports, module.filename, module.id, module.loaded, module.parent, module.require, os.arch, os.constants, os.cpus, os.endianness, os.EOL, os.freemem, os.homedir, os.hostname, os.loadavg, os.networkInterfaces, os.platform, os.release, os.tmpdir, os.totalmem, os.type, os.uptime, os.userInfo, path.basename, path.delimiter, path.dirname, path.extname, path.format, path.isAbsolute, path.join, path.normalize, path.parse, path.posix, path.relative, path.resolve, path.sep, path.win32, process, process.abort, process.arch, process.argv, process.argv0, process.channel, process.chdir, process.config, process.connected, process.cpuUsage, process.cwd, process.disconnect, process.emitWarning, process.env, process.execArgv, process.execPath, process.exit, process.exitCode, process.getegid, process.geteuid, process.getgid, process.getgroups, process.getuid, process.hrtime, process.initgroups, process.kill, process.mainModule, process.memoryUsage, process.nextTick, process.pid, process.platform, process.release, process.send, process.setegid, process.seteuid, process.setgid, process.setgroups, process.setuid, process.stderr, process.stdin, process.stdout, process.title, process.umask, process.uptime, process.version, process.versions, querystring.escape, querystring.parse, querystring.stringify, querystring.unescape, r.clearLine, readable.pause, readable.pipe, readable.push, readable.push, readable.read, readable.read, readable.resume, readable.setEncoding, readable.unpipe, readable.unshift, readable.wrap, readable._read, readStream.bytesRead, readStream.isRaw, readStream.path, readStream.setRawMode, repl.start, request.abort, request.aborted, request.end, request.flushHeaders, request.setNoDelay, request.setSocketKeepAlive, request.setTimeout, request.write, require, require.cache, require.extensions, response.addTrailers, response.end, response.finished, response.getHeader, response.getHeaderNames, response.getHeaders, response.hasHeader, response.headersSent, response.removeHeader, response.sendDate, response.setHeader, response.setTimeout, response.statusCode, response.statusMessage, response.write, response.writeContinue, response.writeHead, rl.clearScreenDown, rl.close, rl.createInterface, rl.cursorTo, rl.emitKeypressEvents, rl.moveCursor, rl.pause, rl.prompt, rl.question, rl.resume, rl.setPrompt, rl.write, script.runInNewContext, script.runInThisContext, server.addContext, server.address, server.address, server.close, server.close, server.connections, server.getTicketKeys, server.listen, server.listen, server.setTicketKeys, server.setTimeout, server.setTimeout, server.timeout, server.timeout, setImmediate, setInterval, setTimeout, socket.addMembership, socket.address, socket.bind, socket.bind, socket.close, socket.dropMembership, socket.ref, socket.send, socket.setBroadcast, socket.setMulticastLoopback, socket.setMulticastTTL, socket.setTTL, socket.unref, stream.Readable, stringDecoder.end, stringDecoder.write, timeout.ref, timeout.unref, tls.connect, tls.createSecureContext, tls.createServer, tls.getCiphers, tlsSocket.address, tlsSocket.authorizationError, tlsSocket.authorized, tlsSocket.encrypted, tlsSocket.getCipher, tlsSocket.getEphemeralKeyInfo, tlsSocket.getPeerCertificate, tlsSocket.getProtocol, tlsSocket.getSession, tlsSocket.getTLSTicket, tlsSocket.localAddress, tlsSocket.localPort, tlsSocket.remoteAddress, tlsSocket.remoteFamily, tlsSocket.remotePort, tlsSocket.renegotiate, tlsSocket.setMaxSendFragment, transform._flush, transform._transform, util.debuglog, util.deprecate, util.format, util.inherits, util.inspect, v8.getHeapStatistics, v8.setFlagsFromString, vm.createContext, vm.isContext, vm.runInContext, vm.runInDebugContext, vm.runInNewContext, vm.runInThisContext, watcher.close, worker.disconnect, worker.exitedAfterDisconnect, worker.id, worker.isConnected, worker.isDead, worker.kill, worker.process, worker.send, worker.suicide, writable.cork, writable.end, writable.setDefaultEncoding, writable.write, writeStream.bytesWritten, writeStream.columns, writeStream.path, writeStream.rows, zlib, zlib.createGunzip, zlib.createGzip, zlib.createInflate, zlib.createInflateRaw, zlib.createUnzip, zlib.deflate, zlib.deflateRaw, zlib.deflateRawSync, zlib.deflateSync, zlib.gunzip, zlib.gunzipSync, zlib.gzip, zlib.gzipSync, zlib.inflate, zlib.inflateRaw, zlib.inflateRawSync, zlib.inflateSync, zlib.unzip, zlib.unzipSync, __dirname, __filename}, % Node.js keywords
    emph={[3] assert, assert.deepEqual, assert.deepStrictEqual, assert.doesNotThrow, assert.equal, assert.fail, assert.ifError, assert.notDeepEqual, assert.notDeepStrictEqual, assert.notEqual, assert.notStrictEqual, assert.ok, assert.strictEqual, assert.throws, describe, toBe, it, xdescribe, beforeEach, afterEach, beforeAll, afterAll, expect, it, xit, xdiscribe, pending, and.callThrough, and.returnValue, and.returnValues, and.callFake, and.throwError, and.stub, .not, .calls.any, .calls.count, .calls.argsFor, .calls.allArgs, .calls.all, .calls.mostRecent, .calls.first, .calls.reset, jasmine.createSpy, jasmine.createSpyObj, jasmine.any, jasmine.anything, jasmine.objectContaining, jasmine.arrayContaining, jasmine.stringMatching, asymmetricMatch,  jasmine.clock, .not.toBeTruthy, .toBeTruthy, .not.toBeFalsy, .toBeFalsy, .not.toBeDefined .toBeDefined, .not.toBeNull .toBeNull, .not.toEqual .toEqual, .not.toBeCloseTo .toBeCloseTo, .not.toContain, .toContain, .not.toMatch, .toMatch, .not.toBeGreaterThan, .toBeGreaterThan, .not.toBeLessThan, .toBeLessThan, .toThrow, .not.toThrow, .toBeNull, .not.toBeNull, .toBeDefined, .not.toBeDefined}, % Node.js Assert, Jasmine, ... keywords
}

  \lstset{
   basicstyle=\normalsize\linespread{1.1}\footnotesize\ttfamily,
   language=JavaScript,
   frame=top,frame=bottom,
   breaklines=true,
   showstringspaces=false,
   tabsize=2,
   upquote = true,
   numbers=left,
   numberstyle=\tiny,
   stepnumber=1,
   numbersep=5pt,
   numberblanklines=false,
   xleftmargin=17pt,
   framexleftmargin=17pt,
   framexrightmargin=17pt,
   framexbottommargin=5pt,
   framextopmargin=5pt,
   alsoother={.},
   captionpos=t,
   literate=
            *{\{}{{\textcolor{antiquefuchsia}{\{}}}{1}% punctuators
            {\}}{{\textcolor{antiquefuchsia}{\}}}}{1}%
            {(}{{\textcolor{antiquefuchsia}{(}}}1%
            {)}{{\textcolor{antiquefuchsia}{)}}}1%
            {[}{{\textcolor{antiquefuchsia}{[}}}1%
            {]}{{\textcolor{antiquefuchsia}{]}}}1%
            {...}{{\textcolor{ballblue}{...}}}1%
            {;}{{\textcolor{antiquefuchsia}{;}}}1%
            {,}{{\textcolor{antiquefuchsia}{,}}}1%
            {>}{{\textcolor{ballblue}{>}}}1%
            {<}{{\textcolor{ballblue}{<}}}1%
            {<=}{{\textcolor{ballblue}{<=}}}1%
            {>=}{{\textcolor{ballblue}{>=}}}1%
            {==}{{\textcolor{ballblue}{==}}}1%
            {!=}{{\textcolor{ballblue}{!=}}}1%
            {===}{{\textcolor{ballblue}{===}}}1%
            {!==}{{\textcolor{ballblue}{!==}}}1%
            {+}{{\textcolor{ballblue}{+}}}1%
            {-}{{\textcolor{ballblue}{-}}}1%
            {*}{{\textcolor{ballblue}{*}}}1%
            {\%}{{\textcolor{ballblue}{\%}}}1%
            {++}{{\textcolor{ballblue}{++}}}1%
            {--}{{\textcolor{ballblue}{--}}}1%
            {<<}{{\textcolor{ballblue}{<<}}}1%
            {>>}{{\textcolor{ballblue}{>>}}}1%
            {>>>}{{\textcolor{ballblue}{>>>}}}1%
            {=}{{\textcolor{ballblue}{=}}}1%
            {&}{{\textcolor{ballblue}{&}}}1%
            {|}{{\textcolor{ballblue}{|}}}1%
            {^}{{\textcolor{ballblue}{^}}}1%
            {!}{{\textcolor{ballblue}{!}}}1%
            {~}{{\textcolor{ballblue}{~}}}1%
            {&&}{{\textcolor{ballblue}{&&}}}1%
            {||}{{\textcolor{ballblue}{||}}}1%
            {?}{{\textcolor{ballblue}{?}}}1%
            {:}{{\textcolor{ballblue}{:}}}1%
            {=}{{\textcolor{ballblue}{=}}}1%
            {+=}{{\textcolor{ballblue}{+=}}}1%
            {-=}{{\textcolor{ballblue}{-=}}}1%
            {*=}{{\textcolor{ballblue}{*=}}}1%
            {\%=}{{\textcolor{ballblue}{\%=}}}1%
            {<<=}{{\textcolor{ballblue}{<<=}}}1%
            {>>=}{{\textcolor{ballblue}{>>=}}}1%
            {>>>=}{{\textcolor{ballblue}{>>>=}}}1%
            {&=}{{\textcolor{ballblue}{&=}}}1%
            {|=}{{\textcolor{ballblue}{|=}}}1%
            {^=}{{\textcolor{ballblue}{^=}}}1%
            {=>}{{\textcolor{ballblue}{=>}}}1%
            {\\b}{{\textcolor{ballblue}{\\b}}}1% escape sequences
            {\\t}{{\textcolor{apricot}{\\t}}}{1}%
            {\\n}{{\textcolor{apricot}{\\n}}}{1}%
            {\\v}{{\textcolor{apricot}{\\v}}}{1}%
            {\\f}{{\textcolor{apricot}{\\f}}}{1}%
            {\\r}{{\textcolor{apricot}{\\r}}}{1}%
            {\\"}{{\textcolor{apricot}{\\"}}}{1}%
            {\\'}{{\textcolor{apricot}{\\'}}}{1}%
            {\\}{{\textcolor{apricot}{\\}}}{1}%
            {0}{{\textcolor{apricot}{0}}}{1}% numbers
            {1}{{\textcolor{apricot}{1}}}{1}%
            {2}{{\textcolor{apricot}{2}}}{1}%
            {3}{{\textcolor{apricot}{3}}}{1}%
            {4}{{\textcolor{apricot}{4}}}{1}%
            {5}{{\textcolor{apricot}{5}}}{1}%
            {6}{{\textcolor{apricot}{6}}}{1}%
            {7}{{\textcolor{apricot}{7}}}{1}%
            {8}{{\textcolor{apricot}{8}}}{1}%
            {9}{{\textcolor{apricot}{9}}}{1}%
            {.0}{{\textcolor{apricot}{.0}}}{2}%
            {.1}{{\textcolor{apricot}{.1}}}{2}%
            {.2}{{\textcolor{apricot}{.2}}}{2}%
            {.3}{{\textcolor{apricot}{.3}}}{2}%
            {.4}{{\textcolor{apricot}{.4}}}{2}%
            {.5}{{\textcolor{apricot}{.5}}}{2}%
            {.6}{{\textcolor{apricot}{.6}}}{2}%
            {.7}{{\textcolor{apricot}{.7}}}{2}%
            {.8}{{\textcolor{apricot}{.8}}}{2}%
            {.9}{{\textcolor{apricot}{.9}}}{2},%
   numberstyle=\normalfont\tiny\textcolor{linenumber} % line number
}

\lstnewenvironment{code}{\lstset{language=JavaScript,inputencoding=latin1,mathescape}}{}
\newcommand{\word}[1]{\texttt{#1}}

\newcommand{\symboldef}[2]{#1\ $\equiv$\ (\ref{#2}; page~\pageref{#2})}

\newcommand{\typurl}{\word{url}}
\newcommand{\typrecord}{\word{record}}
\newcommand{\typmethod}{\word{method}}
\newcommand{\typheader}{\word{header}}
\newcommand{\typstatus}{\word{status}}
\newcommand{\typcursor}{\word{cursor}}
\newcommand{\typcsrf}{\word{csrfToken}}
\newcommand{\typpassword}{\word{password}}
\newcommand{\typIsNotAuth}{\word{isNotAuth}}

\newcommand{\plural}[1]{{{#1}\word{'s}}}
\newcommand{\opt}[1]{?{#1}}
\newcommand{\asyncmap}[2]{{#1}\rightsquigarrow{#2}}
\newcommand{\map}[2]{{#1}\to{#2}}


\begin{document}

\maketitle

\begin{flushright}\it
You are currently reading the source code of the application.\linebreak
Run the processor to build.
\end{flushright}

\chapter{Algebras}

\section{Network calls}

The application being meant to run in a web browser, we need to handle
communication between the application runtime environment and a remote backend
server we own, in order to guarantee information sharing between different
clients (users, but also devices).

This task covers different cross-cutting concerns we need to tackle
correctly.

\subsection{Low-level HTTP bridge}

The native HTTP communication API we exploit is the XMLHttpRequest
API. The following definition is a bridge around it:

\begin{definition}[Http Request Bridge]\label{http_request_bridge}
The native Http Request Bridge is the asynchronous map
	\[
	\asyncmap{
	  \typurl\times\typrecord\times\typmethod\times\plural{\typheader}
	}{
	  \typstatus\times\plural{\typrecord}
	} \]
defined by the code
\begin{code}
({ url, record, headers }) => new Promise((res, rej) => {
  var req = new XMLHttpRequest();
  req.onerror = e => rej(e);
  req.onreadystatechange = () => {
    if (req.readyState == 4)
      res({ status: req.status, data: req.response });
  };
  req.open(method, url);
  for(key in headers) if (headers[key])
    req.setRequestHeader(key, headers[key]);
  req.send(record);
});
\end{code}
\end{definition}

\subsection{Higher-level HTTP bridge}

The previous HTTP bridge is too low-level to reflect part of
our core business. We therefore construct a wrapper around it,
with business-oriented quantities.

\begin{definition}[Business Request Bridge]\label{http_business_request_bridge}
The Business Request Bridge is the map
	\[
	\map{
	  (\typrecord + \typcursor)\times \typpassword\times\typcsrf
	}{
	  \plural{\typheader}
	} \]
defined by the code
\begin{code}
({ record, cursor, password, csrfToken }) => ({
  'Accept'         : 'application/json',
  'Content-Type'   : record ? 'application/json' : undefined,
  'Authentication' : password,
  'X-Csrf-Token'   : csrfToken
 });
\end{code}
\end{definition}

Similarly, we can wrap the Http response with a business-oriented
API that will suit us more:

\begin{definition}[Business Response Bridge]\label{http_business_response_bridge}
The Http Business Response Bridge is the map
	\[ \map{
	  \typstatus
	}{
	  (\asyncmap{\plural{\typrecord}}{\opt{\plural{\typrecord}}})
	}
	\]
defined by the code
\begin{code}
({ status }) => {
  switch(status) {
    case 200: return x => Promise.resolve(x);
    case 401: return x => Promise.resolve(null);
    case 403: return x => Promise.reject("Forbidden");
    default : return x => Promise.reject("Unknown");
  }
};
\end{code}
\end{definition}

Gluing the business bridges around the low-level one,
we infer the following deduction:

\begin{deduction}[Higher-level HTTP Bridge]\label{http_business_bridge}
Given \symboldef{f}{http_request_bridge}, \symboldef{h}{http_business_request_bridge}
and \symboldef{r}{http_business_response_bridge},
the following code defines an asynchronous mapping
	\[
	\asyncmap{
	  \typurl\times(\typrecord + \typcursor)\times
		\typpassword\times\typcsrf
	}{
	  \opt{\plural{\typrecord}}
	}
	\]
\begin{code}
x => !(x.headers = h(x))
     || f(x).then(({ status, records }) => r(status)(records));
\end{code}
\end{deduction}

\section{Records and Events}

The \word{record} type is the basic type that travels through the network,
from frontend to backend (POST request) and from backend to frontend (GET request).

A contrario, the \word{event} type corresponds to what the user will send,
from a business point of view. It corresponds to an appointment in the calendar,
or more generally, to an action on the calendar.

The two notions are related according to the following decomposition:
	\[ \word{record} = \word{event} \times \word{version} \times \word{time} ;\]
where \word{version} is the version number of the event,
and \word{time} is a long number that represents the timestamp (in millis)
at which the server accepted to register the event.

The timestamp is approximately unique, due to the low usage of the application.
However, it will not be used as an identifier, but rather as a cursor
in order to sort the events chronologically.

The version number is low and corresponds to evolution of the data model
in the event type itself: as the application grows and evolves, this
type may change its shape.

\subsection{Version-0 events}

Events versionned with 0 (or unversionned) are events used directly by the
frontend. This model is expected to change from version to version.

Currently, the pattern is
	\[ \word{date} \times \word{time}\times \word{description} \times\word{kind} ,\]
where \word{kind} can take the values \textit{create} or \textit{cancel},
depending on whether the event at timestamp $\word{date}\times\word{time}$
is created or cancelled.

The types \word{date} and \word{time} are expected to be totally
ordered. The \word{record} type, as being the type that travels through the
network, is always concerned with JSON versions of the V0-JavaScript
event definition.

The conversion is handled by the following definitions:
\begin{definition}[Event to record]\label{event_to_record_converter}
The following map defines a converter
	\[ \word{event} \to \word{record} \]
\begin{code}
event => ({
  event: f(event),
  version: 1
});
\end{code}
where \symboldef{f}{event_v1} is the map that defines version-1 events.
\end{definition}

The above definition is likely to change, since it is usually not useful
to keep it if the current event version is greater that 1.

Conversely, however, since the remote end point may still contain older versionned
events, we need a way to convert a record back on a V0 event.
The following definition should be updated according to evolution changes:

\begin{definition}[Version to converter]\label{version_to_record_converter}
Let \symboldef{f1}{event_v1_invert} be the invert map for V1 events.
The following code defines a mapping
	\[ \word{record}\to (\word{record}\to \word{event})\]
\begin{code}
({ version }) => {
  switch (version) {
    case 1:  return f1;
    default: return () => null;
  }
};
\end{code}
\end{definition}

Reducing currying yields:
\begin{deduction}[Record to Event]\label{record_to_event_converter}
Let \symboldef{f}{version_to_record_converter} be the map that associates, to
each version, a converter of \word{record}. The following code defines a map
	\[ \word{record}\to \word{event}\]
\begin{code}
({ event, version }) => f(version)(event);
\end{code}
\end{deduction}

\subsection{Version-1 events}

\begin{definition}[Version-1 Event]\label{event_v1}
Events of version 1 are defined as the range of the mapping
\begin{code}
({ date, time, description, kind }) => ({
  strDate: date.toString(),
  strTime: time.toString(),
  description,
  kind
});
\end{code}
\end{definition}

\begin{definition}[Invert mapping]\label{event_v1_invert}
Let \symboldef{h}{event_v1} be the event v1 definition map, and
let \symboldef{asDate}{as_date}, \symboldef{asTime}{as_time}
be respectively given by the custom date and time algebras (see below).
The following code defines an invert mapping
	\[ \word{eventv1}\to \word{event} \]
\begin{code}
({ strDate, strTime, description, kind }) => {
  convert: {
    if (!kind || !description || !strDate || !strTime)
      break convert;
    var date = asDate(strDate),
        time = asTime(strTime);
    if (!date || !time) break convert;
    if (kind != 'create' || kind != 'cancel') break convert;

    return { kind, description, date, time };
  } or_else_nothing: {
    return null;
  }
};
\end{code}
\end{definition}

\section{Dates and times}

\subsection{String representations}

Dates and times are of crucial importance for the calendar application.
Dates are always thought of as being strings following the pattern
\textit{yyyy-mm-dd}, while times are strings following the pattern
\textit{hh:mm}.

Those definitions are consistent with how HTML input fields of type
date and time, are managing data. It also automatically fixes for us the
question about how to sort, since string-ordering is now equivalent to
date and time ordereing.

We begin with a simple utility function:
\begin{definition}[Number formatter]\label{number_format}
The following formatter can be used to convert non negative numbers to
two-digits-long characters strings:
\begin{code}
x => !(x < 0 : (x = 0)|true : x) ||  x <= 9 ? x : ('0' + x);
\end{code}
\end{definition}

Formatting dates and times is straightforward:

\begin{definition}[String as date]\label{as_date}
Let \symboldef{f}{number_format} be the standard number formatter on
two-digits. The date formatter
	\[ \word{string} \to \word{date} \]
is defined as
\begin{code}
strDate => { try {
  var date = Date.parse (strDate),
         x = [
               date.getFullYear(),
               date.getMonth() + 1,
               date.getDate()
             ].map(f);
  x.toString = function() {
    return this.join('-');
  };
  return x;
} catch(ignored) { return null; } }
\end{code}
\end{definition}

\begin{definition}[String as time]\label{as_time}
Let \symboldef{f}{number_format} be the standard formatter on two-digits.
The time formatter
	\[ \word{string} \to \word{time} \]
is defined by the code
\begin{code}
strTime => { try {
  var x = strTime.split(':').splice(0,2).map(f);
  if (x[0] >= 24 || x[1] >= 60) return null;
  x.toString = function() {
    return this.join(':');
  }
  return x;
} catch(ignored) { return null; } }
\end{code}
\end{definition}

\subsection{Weeks and iterations}

In order to

\end{document}